\begin{sloppypar} % помогает в кириллическом документе выровнять текст по краям
\newpage % Так добавляется  новая страница
\section{РАЗРАБОТКА СТРУКТУРНОЙ СХЕМЫ УСТРОЙСТВА} %Объявили начало раздела
Структурная схема устройства представлена на рисунке \ref{ris:Figures/struct.png}.

\imghh{150mm}{Figures/struct.png}{Структурная схема устройства}
Стоит отметить, что измерять ток цифровые устройства не умеют, поэтому, ток преобразуют в напряжение, чтобы АЦП мог оцифровать его. 
Принцип работы устройства заключается в следующем. АЦП имеет два канала. На один канал подключен выход инструментального уселителя, усиливающего напряжение на низкоомном шунте.  Данный канал используется для измерения токов в диапазоне 1мА-100мА. Для измерения токов в диапазоне 1мкА-1мА используется схема трансимпедансного усилителя\cite {CIRCUIT-CELLAR}, изображенная на рисунке \ref{ris:Figures/struct-tr.png}.
\imghh{80mm}{Figures/struct-tr.png}{Трансимпедансный усилитель}

Измеренное значение напряжения пересчитывается в ток, и выводится на IPS дисплей. Так же, результаты могут быть переданы по интерфейсу CAN с фиксированной скоростью.

Блок питания формирует напряжение +12В и -12В из 220В для питания операционных усилителей. Посредством использования стабилизаторов напряжения из 12В получаем напряжения в 3.3В и 5В, необходимые для питания микроконтроллера и других элементов схемы.


\end{sloppypar}
