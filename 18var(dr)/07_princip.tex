\begin{sloppypar} % помогает в кириллическом документе выровнять текст по краям
\newpage % Так добавляется  новая страница
\section{РАЗРАБОТКА ПРИНЦИПИАЛЬНОЙ СХЕМЫ УСТРОЙСТВА} %Объявили начало раздела
Электрическая принципиальная схема представлена в приложении.
\subsection{Разработка аналоговой части}
Основным требованием для ОУ в схеме трансимпендасного усилителя является маые входные токи - они должны быть меньше, чем минимальная разрешенная погрешность измерения. Данному требованию удовлетворяет AD8603 от Analog Devices.
Его основные особенности представлены на рисунке \ref{ris:Figures/AD8603.png}.
\imghh{90mm}{Figures/AD8603.png}{Особенности AD8603}


\subsection{Выбор микроконтроллера}
Серия	stm32 f1	
Ядро	arm cortex-m3	
Ширина шины данных	32-бит	
Тактовая частота	24 мгц	
Количество входов/выходов	37	
Объем памяти программ	64 кбайт(64k x 8)	
Тип памяти программ	flash	
Объем RAM	8k x 8	
Наличие АЦП/ЦАП	ацп 10x12b/цап 2x12b	
Встроенные интерфейсы	i2c, irda, lin, spi, uart	
Встроенная периферия	dma, pdr, por, pvd, pwm, tempsensor, wdt	
Напряжение питания	2…3.6 в	
Рабочая температура	-40…+85c	
Корпус	lqfp-48(7x7)	
Вес, г	1.4



\subsection{Выбор трансивера для CAN}






MCP2551-I/SN \cite{CAN}является высокоскоростным приемопередатчиком  CAN, стойким к ошибкам устройством, которое служит в качестве интерфейса между контроллером CAN протокола и физической шиной.
MCP2551-I/SN создает возможности дифференциальной передачи и приема для CAN контроллера и полностью совместим со стандартом ISO-11898.
Приемопередатчик рекомендован для использования в системах с напряжением питания 12 В и 24 В.
- Скорость передачи данных до 1Мбит/с
- Управление выборкой данных для уменьшения влияния электромагнитных помех
- Сброс по включению питания и снижению напряжения питания
- Сброс MCP2551-I/SN не влияет на текущий обмен данными на шине CAN - Низкое энергопотребление в режиме ожидания
- Защита от электрических импульсов
- Защита от кратковременного подключения к цепям питания системы
- Подключение до 112 узлов на одну CAN шину





% Cхема включения приведена в даташите, представлена на рисунке \ref{ris:Figures/AD8603_on.png}.
% \imghh{150mm}{Figures/AD8603_on.png}{Включение AD8603}


\subsection{Выбор индикатора}

В качестве индикатора был выбран Дисплей TFT IPS 80x160 0.96" SPI RGB
Полноцветный дисплей на IPS матрице с контроллером ST7735S

Технические характеристики:
Разрешение: 80х160


Количество цветов: 65000


Угол обзора: > 160°


Напряжение питания: 3.3 В


Потребление: 0.04 ватта


Драйвер: ST7735S


Интерфейс: SPI


Размер дисплея: 21.7 х 10.8 мм.


Габаритные размеры: 30 х 24 х 4.1 мм







\end{sloppypar}
