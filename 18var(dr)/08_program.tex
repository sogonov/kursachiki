\begin{sloppypar} % помогает в кириллическом документе выровнять текст по краям
\newpage % Так добавляется  новая страница
\section{РАЗРАБОТКА ПРОГРАММЫ} %Объявили начало раздела
\subsection{Разработка алгоритма}

Проанализируем задание, учитывая ранее описанное. Необходимо получать данные от АЦП ADS1115 по интерфейсу I2C. Так же данные передаются по интерфейсу CAN.


Для работы программы необходимо для начала разработать алгоритм. Алгоритм нашего устройства представлен на рисунке \ref{ris:prog/algo.png}.

\imghh{70mm}{prog/algo.png}{Алгоритм работы устройства}


% Главное тело программы работает так -  инициализирует всю необходимую переферию микроконтроллера, после чего проверяет launch\_flg - флаг, который поднимается в прерывании от таймера каждые 2.5мс, что соответствует частоте обновления данных в 400Гц. 


% Так же есть три прерывания -- по переполнению счетчика таймера, и два внешних - по изменению уровня на выводах акселерометра INT0 INT1, подключенных к выводам микроконтроллера PA4 и PA5.


% Если флаг запуска  launch\_flg==true, то ждем, когда установится флаг data\_flg(который устанавливается по готовности данных в акселерометре). После этого проверяем флаг свободного падения free\_fall\_flg==true. Если он активен - то помимо данных об ускорении пишем еще и сообщение о том, что произошло падение. Если free\_fall\_flg==false, то просто передаем данные по ускорению. Таким образом, для определения характера движения была использована гибкая система прерываний ADXL345. Подробнее о ней в следующем разделе.




\subsection{Разработка кода}

\subsubsection{Выбор программного обеспечения}
Для разработки ПО под STM32 можно использовать различные IDE. 

Какие плюсы у данного ПО: абсолютно бесплатно, нет ограничения по размеру кода, есть неплохой отладчик, простая установка и настройка. Так же, стоит отметить, что данная платформа кроссплатформенная - есть версии для Windows, Linux и даже MacOS. Ознакомиться с STM32CubeIDE можно в \cite{STM32CubeIDE}

\subsubsection{Инициализация периферии}
В STM32CubeIDE встроен STM32CubeMx -- программный продукт, позволяющий при помощи достаточно понятного графического интерфейса произвести настройку любой имеющейся на борту микроконтроллера периферии. Подробнее об этом можно прочитать в \cite{cube}

Сначала в настройках Reset and Clock Controller(RCC) подключаем кварцевые резонаторы, как показано на рисунке \ref{ris:cub/rcc.png}.
\imghh{150mm}{cub/rcc.png}{Настройки RCC}

Затем подключим порты ввода-вывода и настроим их как внешний источник прерываний, как показано на рисунке \ref{ris:cub/gpio.png}.
\imghh{150mm}{cub/gpio.png}{Настройки портов ввода-вывода}

Затем подключим интерфейс СAN, как показано на рисунке \ref{ris:cub/spi.png}.
\imghh{150mm}{cub/spi.png}{Настройки портов ввода-вывода}

После этого можно настроить тактирование на вкладке Clock Configuration, как показано на рисунке \ref{ris:cub/clock.png}.
\imghh{150mm}{cub/clock.png}{Настройки тактирования}

\end{sloppypar}
