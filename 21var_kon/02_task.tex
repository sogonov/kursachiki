\begin{sloppypar} % помогает в кириллическом документе выровнять текст по краям
\newpage % Так добавляется  новая страница
\section*{ЗАДАНИЕ} %Объявили начало раздела

Разработать усилитель ЭГС с гальванической развязкой. Элемент развязки – оптрон. Вид модуляции ШИМ. Предусмотреть защиту от помех электрохирургического инструмента и индикатор плохого контакта.


Исходные данные:
\begin{itemize}

	\item[--]Значение ёмкости между силовой линии и телом пациента: С=5 пФ;
	\item[--]Значение ёмкости между телом пациента и землёй: С1 = 200 пФ; 

	\item[--]Диапазон изменения сопротивлений электродов: \delta  Z = 10--100 кОм;

	\item[--]Погрешность измерения во входной цепи: \beta = 0,4\%;
	\item[--]Разность электродных потенциалов: \Delta U = 200 мВ;
			\item[--]Диапазон входных напряжений:\begin{math} U_\textup{вх} = 0,01--0,5 мВ; \end{math}
		\item[--]Полоса пропускания усилителя: \Delta F = 0,01 − 10 Гц;
		
	\item[--]Неравномерность АЧХ в полосе пропускания: \delta = ±5\%;
	\item[--]Диапазон выходных напряжений:\begin{math} U_\textup{вых} = ±10 \textup{В};\end{math}
	\item[--]Напряжение внутренних шумов, приведенных ко входу: \begin{math}U_\textup{ш} = 15 мкВ;\end{math}
	\item[--]Амплитуда помехи от силовой сети на выходе: \begin{math}U_\textup{п} = 150 мВ;\end{math}
	\item[--]Длина кабеля отведений: L = 2,5 м;
	\item[--]Емкость кабеля на единицу длины: \begin{math}C_\textup{к} = 20 \textup{пФ}\/\textup{м};\end{math}
	
	
	
	
	\item[--]Емкость изоляции: С_\textup{из}  = 15 пФ;
	\item[--]Сопротивление изоляции: С_\textup{из} = 10×10^{10}  Ом.


\end{itemize}
\end{sloppypar}







