\begin{sloppypar} % помогает в кириллическом документе выровнять текст по краям
\newpage % Так добавляется  новая страница
\section{РАЗРАБОТКА СТРУКТУРНОЙ СХЕМЫ УСТРОЙСТВА} %Объявили начало раздела
Структурная схема устройства представлена на рисунке \ref{ris:Figures/struct.png}.
\imghh{150mm}{Figures/struct.png}{Структурная схема устройства}

Принцип работы устройства основан на регистрации разности биопотенциалов, возникающей под действием электрической активности сердца. Система электродов позволяет эту разность измерить. Сигнал будет содержать помимо полезной составляющей синфазную помеху. Для ее устранения используется драйвер нейтрального электрода. Сигнал с электродов с 1 по 3 усредняется, инвертируется в цепи обратной связи и подается на тело человека через электрод 4, тем самым подавляя синфазную составляющую. Так как разность биопотенциалов очень мала по своему значению, то для дальнейшего анализа ее необходимо усилить. Для этого следующим шагом служит усилитель биопотенциалов. Усиленный сигнал преобразуется в цифровой посредством АЦП, проходит через фильтр и поступает на микроконтроллер для обработки. Так же с микроконтроллера данные передаются по Bluetooth, передача данных осуществляется по таймеру. Предусмотрена запись данных на карту памяти. Система электродов, драйвер нейтрального электрода, усилитель, АЦП и цифровой фильтр интегрированы в AFE микросхему ADS1293.
Эти данные поступают в микроконтроллер, где они проходят первичную обработку, 

Так же, данные передаются по модулю Bluetooth, интегрированному в микроконтроллер. Передача данных запускается по таймеру. На устройстве есть LED-индикатор, который сигнализирует о передаче пакета данных. 

Все элементы схемы питаются от литий-полимерного аккумулятора, имеющего номинальное напряжение 3.7 В, и DC-DC преобразователя, встроенного в микроконтроллер, который стабилизирует напряжение до уровня 3.3 В, необходимого всем элементам устройства.

\end{sloppypar}
