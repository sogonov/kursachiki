\begin{sloppypar} % помогает в кириллическом документе выровнять текст по краям
\newpage % Так добавляется  новая страница
\section*{ВВЕДЕНИЕ} %Объявили начало раздела
Автоматизация различных процессов на базе интеллектуальных систем невозможна в современном мире без использования устройств такого типа, как микроконтроллер. Многофункциональные, компактные микроконтроллеры применяются во многих современных приборах, бытовом оборудовании, прочих инженерно-технических объектах, а также в медицинской диагностике.

Согласно статистике сердечно-сосудистые заболеваний являются причиной смерти 17,9 млн человек в год. Именно поэтому мониторинг и диагностика состояния сердечно-сосудистой системы человека является такой важной задачей. Самым простым методом диагностики является ЭКГ.

В данном курсовом проекте рассматривается проектирование носимого монитора ЭКГ, автономной системы, позволяющий вести непрерывный мониторинг показателей сердечно-сосудистой системы человека. В процессе проектирования были выбраны микросхема ADS1293 и микроконтроллер STM32WB55RCV6 со встроенным модулем Bluetooth, а также была написана программа управления на языке Си.

\end{sloppypar}
