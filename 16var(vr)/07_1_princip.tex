\subsection{Разработка( аналоговой части}

Для получения кардиосигнала возможны несколько подходов - проектирование собственных схемотехнических решений на основе дискретных компонентов, или использование \acf{AFE} микросхем, например, ADS1293  от Texas Instruments.

Микросхема ADS1293 предназначена для измерения биопотенциалов, в таких медицинских приборах, как портативные электрокардиографы с батарейным питанием, холтеровские мониторы и аппаратура беспроводного мониторинга пациентов. \cite{ADS1293}

ADS1293 способна поддерживать от одного до пяти каналов, что позволяет существенно сократить габариты, энергопотребление и полную стоимость масштабируемых измерительных медицинских систем. Каждый канал ADS1293 может быть независимо запрограммирован на работу со специальными (отличными от других) частотой выборки и полосой пропускания. 

На основании анализа функциональных возможностей и технических характеристик \ac{AFE} ADS1293, можно сделать вывод о том, что кроме очевидных преимуществ по габаритам в сравнении с аналоговой частью холтеровских мониторов на дискретных операционных усилителях (\acs{OU}) и аналого-цифровых преобразователях (\acs{АЦП}), гибридная \ac{IS} обладает достаточно низким энергопотреблением даже в активном режиме, высоким соотношением сигнал/шум и достаточным динамическим диапазоном для решения задач холтеровского мониторирования. Как следствие высокой интеграции аналоговой части и \ac{АЦП}, а так же цифровой подсистемы первичной обработки квантованного сигнала использование \ac{AFE} позволило существенно упростить схемные решения монитора \acs{ЭКГ}, уменьшить габариты и продолжительность автономной работы от батареи той же емкости. 





Структурная схема акселерометра из даташита ADS1293 \cite{DS1293} приведена на рисунке \ref{ris:Figures/ads1293.png}.

\imghh{160mm}{Figures/ads1293.png}{Структурная схема ADS1293}

Оцифрованный аналоговый сигнал от ADS1293 передается микроконтроллеру поcредством \acf{SPI}.

% Схема подключения представлена на рисунке \ref{ris:Figures/spi.png}.


% \imghh{100mm}{Figures/spi.png}{Cхема подключения ADS1293 к микроконтроллеру по SPI}}


Так же, в даташите приведена рекомендованная схема включения для трехэлектродной схемы(рисунок \ref{ris:Figures/ads1293_connect.png}).

\imghh{160mm}{Figures/ads1293_connect.png}{Трехэлектродная схема включения}



Нумерация и назначение выводов ADS1293 приведены ниже (рисунки \ref{ris:Figures/ads1293_io.png}, \ref{ris:Figures/ads1293_io2.png}).
\imghh{80mm}{Figures/ads1293_io.png}{Нумерация выводов}
\imghh{150mm}{Figures/ads1293_io2.png}{Назначение выводов}




