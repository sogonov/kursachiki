\begin{sloppypar} % помогает в кириллическом документе выровнять текст по краям
\newpage % Так добавляется  новая страница
\section{РАЗРАБОТКА ПРИНЦИПИАЛЬНОЙ СХЕМЫ УСТРОЙСТВА} %Объявили начало раздела
Электрическая принципиальная схема представлена в приложении.

\subsection{Разработка(надо придумать заголовок)}

Для получения кардиосигнала возможны несколько подходов - проектирование собственных схемотехнических решений на основе дискретных компонентов, или использование \acf{AFE} микросхем, например, ADS1293  от Texas Instruments.

Микросхема ADS1293 предназначена для измерения биопотенциалов, в таких медицинских приборах, как портативные электрокардиографы с батарейным питанием, холтеровские мониторы и аппаратура беспроводного мониторинга пациентов. \cite{ADS1293}

ADS1293 способна поддерживать от одного до пяти каналов, что позволяет существенно сократить габариты, энергопотребление и полную стоимость масштабируемых измерительных медицинских систем. Каждый канал ADS1293 может быть независимо запрограммирован на работу со специальными (отличными от других) частотой выборки и полосой пропускания. 

На основании анализа функциональных возможностей и технических характеристик \ac{AFE} ADS1293, можно сделать вывод о том, что кроме очевидных преимуществ по габаритам в сравнении с аналоговой частью холтеровских мониторов на дискретных операционных усилителях (\acs{OU}) и аналого-цифровых преобразователях (\acs{АЦП}), гибридная \ac{IS} обладает достаточно низким энергопотреблением даже в активном режиме, высоким соотношением сигнал/шум и достаточным динамическим диапазоном для решения задач холтеровского мониторирования. Как следствие высокой интеграции аналоговой части и \ac{АЦП}, а так же цифровой подсистемы первичной обработки квантованного сигнала использование \ac{AFE} позволило существенно упростить схемные решения монитора \acs{ЭКГ}, уменьшить габариты и продолжительность автономной работы от батареи той же емкости. 




















Структурная схема акселерометра из даташита ADS1293 \cite{DS1293} приведена на рисунке \ref{ris:Figures/ads1293.png}.

\imghh{160mm}{Figures/ads1293.png}{Структурная схема ADS1293}

Оцифрованный аналоговый сигнал от ADS1293 передается микроконтроллеру поcредством \acf{SPI}.

% Схема подключения представлена на рисунке \ref{ris:Figures/spi.png}.


% \imghh{100mm}{Figures/spi.png}{Cхема подключения ADS1293 к микроконтроллеру по SPI}}


Так же, в даташите приведена рекомендованная схема включения для трехэлектродной схемы(рисунок \ref{ris:Figures/ads1293_connect.png}).

\imghh{160mm}{Figures/ads1293_connect.png}{Трехэлектродная схема включения}



Нумерация и назначение выводов ADS1293 приведены ниже (рисунки \ref{ris:Figures/ads1293_io.png}, \ref{ris:Figures/ads1293_io2.png}).
\imghh{90mm}{Figures/ads1293_io.png}{Нумерация выводов}
\imghh{160mm}{Figures/ads1293_io2.png}{Назначение выводов}




 
% здесь налить водички про то почему был выбрана ADS1293


\subsection{Выбор микроконтроллера}
С учетом технического задания микроконтроллер должен обладать следующими свойствами:
\begin{onehalfspace}
	\begin{itemize}
		\item[--]Интерфейс для работы с микросхемой ADS1293: SPI;
		\item[--]Интерфейс для работы с внешней флеш-памятью: SPI или $I^2$C;
		\item[--]Для передачи данных по Bluetooth: встроенный стек протокола Bluetooth;
		\item[--]Малое энергопотребление;
		\item[--]Свободные выводы для подключения индикатора и выводов прерываний от ADS1293;
	\end{itemize}
\end{onehalfspace}

Для решения задачи был выбран микроконтроллер STM32WB55RCV6 фирмы ST Microelectronics \cite {STM}.STM32WB55 содержит два производительных ядра ARM-Cortex:
\begin{onehalfspace}
	\begin{itemize}
		\item[--] ядро ARM® -Cortex® M4 (прикладное), работающее на частотах до 64 МГц, для пользовательских задач имеется модуль управления памятью, модуль плавающей точки, инструкции ЦОС (цифровой обработки сигналов), графический ускоритель (ART accelerator);
		\item[--] ядро ARM®-Cortex® M0+ (радиоконтроллер) с тактовой частотой 32 МГц, управляющее радиотрактом и реализующее низкоуровневые функции сетевых протоколов;
	\end{itemize}
\end{onehalfspace}

Данный микроконтроллер включает в себя все необходимые периферийные устройства, такие как интерфейсы передачи данных SPI,необходимый для подключения к акселерометру, и радиоконтроллер с поддержкой Bluetooth.

Основные характеристики:
\begin{onehalfspace}
	\begin{itemize}
		\item[--] типовое энергопотребление 50 мкА/МГц (при напряжении питания 3 В);
		\item[--] потребление в режиме останова 1,8 мкА (радиочасть в режиме ожидания (standby));
		\item[--] потребление в выключенном состоянии (Shutdown) менее 50 нА;
		\item[--] диапазон допустимых напряжений питания 1,7…3,6 В (встроенный DC-DC–преобразователь и LDO-стабилизатор);
		\item[--] рабочий температурный диапазон -40…105°С.
	\end{itemize}
\end{onehalfspace}


Структурная схема микроконтроллера приведена на рисунке \ref{ris:Figures/stm32.png}, а назначение выводов портов корпуса на рисунке \ref{ris:Figures/stm32io.png}.

\imghh{90mm}{Figures/stm32io.png}{Назначение выводов}


\imghh{110mm}{Figures/stm32.png}{Структурная схема}


Подключение будет осуществляться согласно типовой схеме из Application note\cite {STM_an}(рисунок \ref{ris:Figures/stm32an.png}).
\imghh{160mm}{Figures/stm32an.png}{Типовая схема подключения STM32WB55}


\subsection{Выбор аккумулятора}
Питание схемы будет осуществляться с помощью аккумулятора LP-310-233350 \cite {li-pol}.
Аккумулятор литий-полимерный LP-310-233350 имеет номинальную емкость 310 мАч, номинальное напряжение 3,7 В, вес 8г. Длина: 50±1 мм. Ширина: 33±1 мм. Толщина: 2,3±1 мм. 

Так как в микроконтроллере есть встроенный DC-DC преобразователь, внешний можно не ставить, напрямую подключив положительный электрод аккумулятора к выводу микроконтроллера ''VBAT''. Внутренний DC-DC преобразователь стабилизирует напряжение на выводе ''VDD'' на уровне 3.3В, необходимого для питания всех элементов схемы.


\end{sloppypar}
