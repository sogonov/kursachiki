\begin{sloppypar} % помогает в кириллическом документе выровнять текст по краям
\newpage % Так добавляется  новая страница
\section{РАЗРАБОТКА ПРИНЦИПИАЛЬНОЙ СХЕМЫ УСТРОЙСТВА} %Объявили начало раздела
Электрическая принципиальная схема представлена в приложении.

\subsection{Выбор акселерометра}
Стоит выяснить, как работают и устроены акселерометры. \cite{MEMS} Это датчики движения, входным сигналом которых является скорость и ускорение объекта. Отличительной особенностью данных устройств является их компактность и стоимость за счет налаженного производства микроэлектромеханических систем (МЭМС).

Основное применение датчики движения нашли в промышленности, а именно в авиации для определения положения летающего аппарата в пространстве и в строительстве. В медицине датчики движения используются редко, однако некоторые методики включают использование акселерометров.

Современные МЭМ акселерометры разделяют по физическому принципу детектирования ускорений, однако широкое распространение получили только 3 вида:

\begin{itemize}

	\item[--]Пьезоэлектрические, основой которых является пъезокристалл. Деформации кристалл приводят к появлению на нем разности потенциалов. Такие акселерометры имеют широкий диапазон частот и выдерживают значительные нагрузки. Однако пьезоэффект возникает только в момент деформации, что не позволяет измерять статические ускорения наподобие гравитационного. Также пьезоэлектрические акселерометры из-за значительного сопротивления пьезокристалла и малой разности потенциалов при деформации требуют высокоомного соединения со схемой. 

	\item[--]Пьезорезистивные своими характеристиками не сильно отличаются от ПЭА, имея столь же малую термостабильность и стабильность смещения. Однако получение полезного электрического сигнала происходит на мостовой схеме с пьезорезистивными элементами, при этом нет необходимости использования высокоомного подключения. Также присутствует возможность самотестирования акселерометров и измерения статических нагрузок 

	\item[--]Емкостные – самый распространенный вид МЭМ акселерометров. Принцип действия заключается в измерении реакции измерительной ячейки, состоящей из сложного конденсатора с переменной емкостью на зондирующий сигнал. При измерении ускорения инерционная масса двигает нестатичную обкладку конденсатора, вследствие чего меняется емкость. При этом емкостные конденсаторы не имеют проблем, связанных с природой пьезоэффекта, а именно имеют конструкторскую легкость при подключении в цепь и возможность самотестирования. Также основными преимуществами является высокая термостабильность. Недостатком можно назвать сложность конструкции, однако при налаженном производстве это фактор не оказывает значительного влияния.

\end{itemize}

Таким образом, современные малогабаритные измерительные модули целесообразно конструировать с емкостными акселерометрами, за счет их стабильности отсутствия требований в схемах высокоомного подключения.


Согласно техническому заданию нам необходим акселерометр с диапазоном регистрируемых ускорений от 2g до 8g и возможностью выдачи показаний с частотой 400 Гц. Данным требованиям соответствует 3-осевой акселерометр ADXL345 \cite {ADXL}, его основные характеристики представлены ниже.


\begin{onehalfspace}
\begin{itemize}
	\item[--] Тип датчика: цифровой, емкостной;
	\item[--]Диапазон регистрируемых ускорений ±2g, ±4g, ±8g, ±16g;
	\item[--]Частота обновления показаний: задается пользователем в диапазоне 0.1-3200 Гц;
	\item[--]Сверхнизкое потребление:  23 мкA в режиме преобразования и 0.1 мкA в режиме ожидания;
	\item[--] напряжение питания: 2-3.6В;
	\item[--] Интерфейс цифрового вывода: $I^2$C, SPI; 
	\item[--] Разрядность: настраиваемая пользователем -- 10 бит в диапазоне ±2g, 13 бит в остальных диапазонах.	
\end{itemize}
\end{onehalfspace}




Структурная схема акселерометра из даташита ADXL345 приведена на рисунке \ref{ris:Figures/accel.png}.

\imghh{160mm}{Figures/accel.png}{Структурная схема акселерометра}

Видно, что устройство состоит из 3-осевого ''сенсора'',  представляющего собой несколько конденсаторов с нестатичными обкладками, ''чувствительной электроники'', аналого-цифрового преобразователя, цифрового фильтра, буфера FIFO для временного хранения результатов преобразования, контроллера питания и логического устройства, контролирующего работу акселерометра и логику прерываний. Устройство содержит выводы данных, соответствующие интерфейсам $I^2$C и SPI. Для связи с акселерометром мы будем использовать $I^2$C. Схема подключения представлена на рисунке \ref{ris:Figures/i2c.png}.


\imghh{100mm}{Figures/i2c.png}{Cхема подключения акселерометра к микроконтроллеру по $I^2$C}


Как видно из рисунка \ref{ris:Figures/i2c.png}, для активации интерфейса $I^2$C необходимо подтянуть вывод $\overline{CS}$ к питанию.

Так же, в даташите приведена рекомендованная для минимизации шумов схема включения акселерометра(рисунок \ref{ris:Figures/accel_connect.png}).

\imghh{100mm}{Figures/accel_connect.png}{Типовая схема включения акселерометра}
Нумерация и назначение выводов ADXL345 приведено ниже (рисунки \ref{ris:Figures/accel_io.png}, \ref{ris:Figures/accel_io2.png}).
\imghh{90mm}{Figures/accel_io.png}{Нумерация выводов}
\imghh{160mm}{Figures/accel_io2.png}{Назначение выводов}




\subsection{Выбор микроконтроллера}
С учетом технического задания микроконтроллер должен обладать следующими свойствами:

\begin{onehalfspace}
	\begin{itemize}
		\item[--]Интерфейс для работы с микросхемой акселерометра: SPI или $I^2$C;
		\item[--]Для передачи данных по Bluetooth: встроенный стек протокола Bluetooth;
		\item[--]Малое энергопотребление;
		\item[--]Свободные выводы для подключения индикатора и выводов прерываний от акселерометра;
	\end{itemize}
\end{onehalfspace}

Для решения задачи был выбран микроконтроллер STM32WB35CCU6A фирмы ST Microelectronics \cite {STM}.STM32WB35 содержит два производительных ядра ARM-Cortex:
\begin{onehalfspace}
	\begin{itemize}
		\item[--] ядро ARM® -Cortex® M4 (прикладное), работающее на частотах до 64 МГц, для пользовательских задач имеется модуль управления памятью, модуль плавающей точки, инструкции ЦОС (цифровой обработки сигналов), графический ускоритель (ART accelerator);
		\item[--] ядро ARM®-Cortex® M0+ (радиоконтроллер) с тактовой частотой 32 МГц, управляющее радиотрактом и реализующее низкоуровневые функции сетевых протоколов;
	\end{itemize}
\end{onehalfspace}
Данный микроконтроллер включает в себя все необходимые периферийные устройства, такие как интерфейсы передачи данных $I^2$C,необходимый для подключения к акселерометру, и радиомодуль с поддержкой Bluetooth, диапазон питающего напряжения от 2 до 3,6 В. 
Основные характеристики:
\begin{onehalfspace}
	\begin{itemize}
		\item[--] типовое энергопотребление 50 мкА/МГц (при напряжении питания 3 В);
		\item[--] потребление в режиме останова 1,8 мкА (радиочасть в режиме ожидания (standby));
		\item[--] потребление в выключенном состоянии (Shutdown) менее 50 нА;
		\item[--] диапазон допустимых напряжений питания 1,7…3,6 В (встроенный DC-DC–преобразователь и LDO-стабилизатор);
		\item[--] рабочий температурный диапазон -40…105°С.
	\end{itemize}
\end{onehalfspace}
Структурная схема микроконтроллера приведена на рисунке \ref{ris:Figures/stm32.png}, а назначение выводов портов корпуса на рисунке \ref{ris:Figures/stm32io.png}

\imghh{160mm}{Figures/stm32.png}{Структурная схема}
\imghh{100mm}{Figures/stm32io.png}{Назначение выводов}

Подключение будет осуществляться согласно типовой схеме из Application note\cite {STM_an}(рисунок \ref{ris:Figures/stm32an.png})
\imghh{160mm}{Figures/stm32an.png}{Типовая схема подключения STM32WB35}



\subsection{Блок питания}

Блок питания будет состоять из аккумулятора LP-130-232635 \cite {li-pol} и DC-DC преобразователя LM3671 \cite {dc-dc}.
Аккумулятор литий-полимерный LP-130-232635 имеет номинальную емкость 130 мАч, номинальное напряжение 3,7 В, вес 3г. Длина: 35±1 мм. Ширина: 26±1 мм. Толщина: 2,3±1 мм. 



DC-DC преобразователь LM3671MF с фиксированным выходным напряжением 3,3 В. Типичный ток покоя 16 мкA, типичный ток в выключенном состоянии - 0.01 мкA, максимальная нагрузка по току 600 мА.

Подключение DC-DC преобразователя будет будет осуществляться согласно типовой схеме из Data Sheet \cite {dc-dc} (рисунок \ref{ris:Figures/dc-dc.png})
\imghh{160mm}{Figures/dc-dc.png}{Типовая схема включения DC-DC–преобразователя}











\end{sloppypar}
