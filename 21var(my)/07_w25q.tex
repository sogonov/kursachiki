\begin{sloppypar} % помогает в кириллическом документе выровнять текст по краям


\subsection{Выбор встроенного носителя}
 В качестве носителя информации выберем последовательную FLASH-память серии W25Q \cite{W25Q}. Данная последовательная память может быть различной ёмкости — 8, 16, 32, 64, 128, 256 Мбит и т. д. Подключается такая память по интерфейсу SPI, а также по многопроводным интерфейсам Dual SPI, Quad SPI и QPI. Мы подключим данную микросхему по обычному интерфейсу SPI.
Необходимый объем памяти для наших нужд рассчитывается по формуле: объем памяти= 16*1*60*24 = 23040bit \approx 23 Mbit
где 16 - необходимый объем памяти для хранения одного измерения в битах, 1- число измерений в минуту, 60 - количество минут в одном часе, 24 - количество часов в сутках.
Таким образом, для хранения данных в течение 24 часов нам нужно выбрать FLASH-память с ёмкостью 32 Mbit.

 
 
 

Краткие основные характеристики W25Q:

\begin{onehalfspace}
	\begin{itemize}
		\item[--]Потребляемая мощность и температурный диапазон:
		\item[--]Напряжение питания 2.7…3.6 В
		\item[--]Типичный потребляемый ток: 4 мА (активный режим), <1 мкА (в режиме снижения мощности)
		\item[--]Рабочий температурный диапазон -40°C…+85°C.
	\end{itemize}
\end{onehalfspace}

Гибкая архитектура с секторами размером 4 кбайт:
\begin{onehalfspace}
	\begin{itemize}
		\item[--]Посекторное стирание (размер каждого сектора 4 кбайт)
		\item[--]Программирование от 1 до 256 байт
		\item[--]До 100 тыс. циклов стирания/записи
		\item[--] 20-летнее хранение данных
	\end{itemize}
\end{onehalfspace}


Максимальная частота работы микросхемы:
\begin{onehalfspace}
	\begin{itemize}
		\item[--]104 МГц в режиме SPI
		\item[--]208/416 МГц — Dual / Quad SPI
	\end{itemize}
\end{onehalfspace}

Также микросхема существует в различных корпусах, но в большинстве случаев распространён корпус SMD SO8. Распиновка микросхемы следующая(рисунок \ref{ris:Figures/w25io.png}).
\imghh{90mm}{Figures/w25io.png}{Распиновка W25Q32}

Описание выводов из  \cite {W25Q}(рисунок \ref{ris:Figures/w25qan.png}).
\imghh{160mm}{Figures/w25qan.png}{Описание выводов W25Q32}
К микроконтроллеру подключается по стандартному интерфейсу SPI.



\end{sloppypar}
