\begin{sloppypar} % помогает в кириллическом документе выровнять текст по краям
\newpage % Так добавляется  новая страница
\section*{ВВЕДЕНИЕ} %Объявили начало раздела
Диабет превратился в одну из основных эпидемий здравоохранения современной эпохи. Ожидается, что во всем мире общее число людей с диабетом вырастет со 171 миллиона в 2000 году до 366 миллионов в 2030 году.\cite{ADXL}

Мало того, что диабет был шестой по значимости причиной смерти, указанной в свидетельствах о смерти в США в 2020 году, но предполагаемая стоимость диабета в Соединенных Штатах в 2002 году составила 132 миллиарда долларов, включая как прямые, так и косвенные расходы (инвалидность, инвалидность, потеря работы, преждевременная смертность).

Известно, что строгий гликемический контроль снижает разрушительные и дорогостоящие вторичные микро- и макрососудистые осложнения, связанные с диабетом, тем самым улучшая качество жизни миллионов пациентов с диабетом и значительно сокращая расходы на здравоохранение. Также, строгий контроль уровня глюкозы обеспечивает клинические преимущества у пациентов в критическом состоянии.


Непрерывный мониторинг глюкозы(НМГ) -- метод регистрации изменений концентрации глюкозы в крови, при котором результаты измерений фиксируются не реже чем каждые 5 мин на протяжении длительного времени (более суток). Применяемые в настоящее время устройства для НМГ позволяют получить данные о гликемии косвенно по концентрации глюкозы в межтканевой жидкости.

В данном курсовом проекте  рассматривается способ создания устройства на базе микроконтроллера, который сможет отслеживать уровень глюкозы в крови человека.  В процессе были подобраны необходимые в задании микроконтроллер с интегрированным модулем Bluetooth, акселерометр, а также написана управляющая программа на языке Си. 



\end{sloppypar}
