\documentclass[a4paper]{article} %размер бумаги устанавливаем А4
% \usepackage[T2A]{fontenc}
\usepackage{polyglossia}
\usepackage{url} % добавляем поддержку url-ссылок
\setmainlanguage[babelshorthands=true]{russian} % Язык по-умолчанию русский с поддержкой приятных команд пакета babel
\setotherlanguage{english} % Дополнительный язык = английский (в американской вариации по-умолчанию)
\newfontfamily{\cyrillicfonttt}{Times New Roman}

\usepackage{minted} %пакет для подсветки кода
\usepackage{color}
\usepackage{xcolor} % to access the named colour LightGray
\definecolor{LightGray}{gray}{0.9}

% для отступа в первом абзаце
\usepackage{indentfirst} 
\parindent=1.25cm % длина отступа в абзацах
% для продвинутых списков
\usepackage{enumitem} 
\usepackage{amssymb,amsfonts,amsmath,cite,enumerate,float,indentfirst} %пакеты расширений
\usepackage{graphicx}% для вставки картинок
\usepackage{url} % добавляем поддержку url-ссылок
\usepackage{hyperref} % пакет для интеграции гиперссылок
\usepackage{amsmath} % добавляем поддержку математических символов
\usepackage{multirow} % понадобится для создания таблицы с объединенными строками
\usepackage{pdfpages}% Добавление внешних pdf файлов
\usepackage{tocloft}
\renewcommand{\cftsecfont}{\mdseries}
\renewcommand{\cftsecpagefont}{\mdseries}
\cftsetindents{section}{0em}{2em}
\cftsetindents{subsection}{0em}{3em}
\cftsetindents{subsubsection}{0em}{4em}
\renewcommand\cfttoctitlefont{\hfill\normalsize\mdseries}
\renewcommand\cftaftertoctitle{\hfill\mbox{}}
\renewcommand{\cftsecleader}{\cftdotfill{\cftdotsep}}

\usepackage{fontspec}
\setmainfont{Times New Roman} %шрифт 
\graphicspath{{images/}}%путь к рисункам
\usepackage[14pt]{extsizes} % для того чтобы задать нестандартный 14-ый размер шрифта

%\makeatletter
%\renewcommand{\@biblabel}[1]{#1.} % Заменяем библиографию с квадратных скобок на точку:
%\makeatother


\usepackage[tableposition=top]{caption}
\usepackage{subcaption}
\DeclareCaptionLabelFormat{gostfigure}{Рисунок #2}
\DeclareCaptionLabelFormat{gosttable}{Таблица #2}
\DeclareCaptionLabelSeparator{gost}{~–~}
\captionsetup{labelsep=gost}
\captionsetup[figure]{labelformat=gostfigure}
\captionsetup[table]{labelformat=gosttable}
\renewcommand{\thesubfigure}{\asbuk{subfigure}}


\makeatletter

\renewcommand{\section}{\@startsection{section}{1}{0pt}%
                                {-3.5ex plus -1ex minus -.2ex}%
                                {2.3ex plus .2ex}%
{\centering\hyphenpenalty=10000\normalfont\normalsize\mdseries}}

\renewcommand{\subsection}{\@startsection{subsection}{1}{0pt}%
                                {-3.5ex plus -1ex minus -.2ex}%
                                {2.3ex plus .2ex}%
{\centering\hyphenpenalty=10000\normalfont\normalsize\mdseries}}
\renewcommand{\subsubsection}{\@startsection{subsubsection}{1}{0pt}%
                                {-3.5ex plus -1ex minus -.2ex}%
                                {2.3ex plus .2ex}%
{\centering\hyphenpenalty=10000\normalfont\normalsize\mdseries}}
\makeatother


\usepackage{setspace}
% \onehalfspacing % полуторный интервал для всего текста
% или \singlespacing % одиночный интервал для всего текста
% или \doublespacing % двойной интервал для всего текста
\setstretch{1.5} % произвольный интервал

% в тексте
% \begin{onehalfspace}
% фрагмент текста с полуторным межстрочным интервалом
% \end{onehalfspace}
% \begin{doublespace}
% фрагмент текста с двойным межстрочным интервалом
% \end{doublespace}

\sloppy % выравнивание по ширине




%%% Поля и разметка страницы %%%
\usepackage{pdflscape}  % Для включения альбомных страниц
\usepackage{geometry}   % Для последующего задания полей
\geometry{left=3cm}% левое поле
\geometry{right=1.5cm}% правое поле
\geometry{top=2cm}% верхнее поле
\geometry{bottom=2cm}% нижнее поле



\renewcommand{\theenumi}{\arabic{enumi}}% Меняем везде перечисления на цифра.цифра
\renewcommand{\labelenumi}{\arabic{enumi}}% Меняем везде перечисления на цифра.цифра
\renewcommand{\theenumii}{.\arabic{enumii}}% Меняем везде перечисления на цифра.цифра
\renewcommand{\labelenumii}{\arabic{enumi}.\arabic{enumii}.}% Меняем везде перечисления на цифра.цифра
\renewcommand{\theenumiii}{.\arabic{enumiii}}% Меняем везде перечисления на цифра.цифра
\renewcommand{\labelenumiii}{\arabic{enumi}.\arabic{enumii}.\arabic{enumiii}.}% Меняем везде перечисления на цифра.цифра



\newcommand{\imgh}[3]{\begin{figure}[h]\center{\includegraphics[width=#1]{#2}}\caption{#3}\label{ris:#2}\end{figure}}
\newcommand{\imghh}[3]{\begin{figure}[H]\center{\includegraphics[width=#1]{#2}}\caption{#3}\label{ris:#2}\end{figure}}


\usepackage{lastpage}
\usepackage{totcount}
\newcounter{totfigures}
\newcounter{tottables}
\newcounter{totreferences}

\makeatletter
    \AtEndDocument{%
      \addtocounter{totfigures}{\value{figure}}%
      \addtocounter{tottables}{\value{table}}%
	  
      \immediate\write\@mainaux{%
        \string\gdef\string\totfig{\number\value{totfigures}}%
        \string\gdef\string\tottab{\number\value{tottables}}%   

      }%
    }
\makeatother

	

\newcommand{\empline}{\mbox{}\newline}
\newcommand{\likechapterheading}[1]{ 
    \begin{center}
   \MakeUppercase{#1}
    \end{center}
    \empline}



\makeatletter
    \renewcommand{\@dotsep}{2}
    \newcommand{\l@likechapter}[2]{{\@dottedtocline{0}{0pt}{0pt}{#1}{#2}}}
\makeatother
\newcommand{\likechapter}[1]{    
    \likechapterheading{#1}    
    \addcontentsline{toc}{likechapter}{\MakeUppercase{#1}}}


\usepackage{cite} % Красивые ссылки на литературу


%% Список литературы с красной строки (без висячего отступа) %%%
\patchcmd{\thebibliography} %может потребовать включения пакета etoolbox
 {\advance\leftmargin\labelsep}
 {\leftmargin=0pt%
  \setlength{\labelsep}{\widthof{\ }}% Управляет длиной отступа после точки
  \itemindent=\parindent%
  \addtolength{\itemindent}{\labelwidth}% Сдвигаем правее на величину номера с точкой
  \advance\itemindent\labelsep%
 }
 {}{}


\makeatletter
\def\@biblabel#1{#1 }
\makeatother


% настройки цветовой палитры для гиперссылок. Цвета можно на свой вкус выбрать здесь:  https://www.overleaf.com/learn/latex/Using_colours_in_LaTeX
\hypersetup{
    citecolor=gray, % цвет цитирования
    colorlinks=true, 
    linkcolor=black, % цвет для гиперссылок 
    filecolor=magenta, % цвет для ссылок на файл      
    urlcolor=mauve} % цвет для url-ссылок
    \usepackage{listings}

\usepackage{color}
\definecolor{dkgreen}{rgb}{0,0.6,0} 
\definecolor{gray}{rgb}{0.3,0.3,0.3}
\definecolor{mauve}{rgb}{0.42,0,0.92}














\begin{document}

\begin{titlepage}
\newpage
\doublespacing
%\linespread{1.3} % полуторный интервал
%\setlength\parindent{1.25cm}
\begin{center}
ФЕДЕРАЛЬНОЕ ГОСУДАРСТВЕННОЕ АВТОНОМНОЕ\\
ОБРАЗОВАТЕЛЬНОЕ УЧРЕЖДЕНИЕ ВЫСШЕГО ОБРАЗОВАНИЯ\\
«САМАРСКИЙ НАЦИОНАЛЬНЫЙ ИССЛЕДОВАТЕЛЬСКИЙ\\
УНИВЕРСИТЕТ ИМЕНИ АКАДЕМИКА С.П. КОРОЛЕВА»	
 \\
\end{center}

\vspace{5em}

\begin{center}
 Институт информатики и кибернетики \\ 
\end{center}

\begin{center}
Кафедра лазерных и биотехнических систем \\ 
\end{center}


\vspace{3em}

\begin{center}
{Отчёт по лабораторной работе № 2\\''Исследование широтно-импульсного модулятора''}
\end{center}

\vspace{11em}



\newbox{\lbox}
\savebox{\lbox}{\hbox{Рожновская Д.О.}}
\newlength{\maxl}
\setlength{\maxl}{\wd\lbox}
\hfill\parbox{7cm}{
\hspace*{4cm}\hspace*{-4cm}Студенты:\hfill\hbox to\maxl{Гуськова К.М.,\hfill}\\
\hspace*{4cm}\hspace*{-4cm}\hfill\hbox to\maxl{Рожновская Д.О.\hfill}\\
\hspace*{4cm}\hspace*{-4cm}\hfill\hbox to\maxl{Cогонов E.A.\hfill}\\
\hspace*{4cm}\hspace*{-4cm}Преподаватель:\hfill\hbox to\maxl{Конюхов В.Н.\hfill }\\
\hspace*{4cm}\hspace*{-4cm}Группа:\hfill\hbox to\maxl{6364-120304D}\\
}

\vspace{\fill}

\begin{center}
Cамара 2023
\end{center}

\end{titlepage}
\setcounter{page}{2}% это титульный лист
\begin{sloppypar} % помогает в кириллическом документе выровнять текст по краям
\newpage % Так добавляется  новая страница
\section*{ЗАДАНИЕ} %Объявили начало раздела

Разработать монитор активности и отслеживания падений  со следующими параметрами: 
\begin{itemize}
	\item[--]Амплитуда сигнала от 0.5 мВ до 4 мВ;
	\item[--]Диапазон частот 0.05 Гц до 40 Гц;
	\item[--]Погрешность регистрации амплитуды и частоты 1\%;
	\item[--]Передача данных по интерфейсу Bluetooth;
	\item[--]Предусмотреть возможность сохранения данных на встроенном носителе в течение суток;
	\item[--]Питание батарейное;
\end{itemize}
\end{sloppypar}% это задание
\begin{sloppypar} % помогает в кириллическом документе выровнять текст по краям
\newpage % Так добавляется  новая страница
\section*{РЕФЕРАТ} %Объявили начало раздела

Пояснительная записка: \pageref*{LastPage}~страниц, \totfig~рисунков, источников, 1 приложение.\\

 % \tottab~таблиц...
МОНИТОР АКТИВНОСТИ И ОТСЛЕЖИВАНИЯ ПАДЕНИЙ, МИКРОКОНТРОЛЛЕР, BLUETOOTH, АКСЕЛЕРОМЕТР, STM32WB, АЛГОРИТМ

% возможно добавим еще каких то фраз


В курсовом проекте разработаны структурная и принципиальная схемы монитора активности и отслеживания падений с датчиком на базе акселерометра, осуществлен выбор микроконтроллера c интегрированным блоком Bluetooth.  Разработан алгоритм анализа данных и программа на языке Си, реализующая его.

\end{sloppypar}
% это реферат
\newpage
\renewcommand{\contentsname}{СОДЕРЖАНИЕ}
{\renewcommand{\baselinestretch}{1.5} %интервал для содержания
\tableofcontents
       
}% это содержание
\begin{sloppypar} % помогает в кириллическом документе выровнять текст по краям
\newpage % Так добавляется  новая страница
\section*{ВВЕДЕНИЕ} %Объявили начало раздела
Падения являются достаточно распространенной проблемой среди людей, в особенности пожилых, которая  наносит существенный вред здоровью. 

Падения являются серьезной проблемой общественного здравоохранения для пожилых людей во всем мире. Отчеты Всемирной организации здравоохранения показывают, что примерно 28-35\% пожилых людей старше 65 лет страдают по крайней мере от одного падения в год, что приводит к травмам мышц или связок, переломам костей и травмам головы. Решением данной проблемы являются носимые детекторы падения. Носимые устройства позволяют осуществлять непрерывный мониторинг независимо от датчиков окружающей среды, что делает их повсеместными системами, которые собирают только пользовательские данные, способствуя расширению возможных сценариев использования. Кроме того, они используют простые датчики (акселерометры и гироскопы) с низким энергопотреблением.

В данном курсовом проекте  рассматривается способ создания устройства на базе микроконтроллера, который сможет отслеживать активность и падения человека.  В процессе были подобраны необходимые в задании микроконтроллер с интегрированным модулем Bluetooth, акселерометр, а также написана управляющая программа на языке Си. 


\end{sloppypar}
% это введение
\begin{sloppypar} % помогает в кириллическом документе выровнять текст по краям
\newpage % Так добавляется  новая страница
\section{РАЗРАБОТКА СТРУКТУРНОЙ СХЕМЫ УСТРОЙСТВА} %Объявили начало раздела

% Многие эксперты и исследователи пытались разработать неинвазивные глюкометры. Однако в настоящее время неинвазивное определение уровня глюкозы в крови все еще имеет проблемы, такие как чувствительность и сигналы фонового шума, которые необходимо преодолеть, а использование защитных покрытий, таких как pHEMA \cite{Ph}, может обеспечить минимально инвазивное и точное измерение изменений уровня глюкозы в крови.

% Таким образом, основной целью данного исследования является достижение точного и эффективного измерения уровня глюкозы в крови у пациентов с диабетом, а также уменьшение боли и дискомфорта во время процесса, что может улучшить качество жизни. Мы разрабатываем электрохимический датчик на основе массива микроигл, модуль схемы обнаружения глюкозы и модуль передачи для размещения в носимом устройстве, которое может непрерывно определять концентрацию глюкозы в интерстициальной жидкости [ISF] с низкой инвазивностью и передавать данные на мобильный телефон по Bluetooth, с точным измерением изменений концентрации глюкозы. Минимально инвазивная рана после использования показана на \ref{ris:Figures/struct2.png}. 
% \imghh{150mm}{Figures/struct2.png}{Минимально инвазивная рана после использования CGMS, эта небольшая рана практически не вызывает боли и кровотечения}


% Концепция микропереноса была применена для переноса глюкозооксидазы на датчик матрицы микроигл. Этот набор микроигл был приобретен у RichHealth Technology. Датчик массива микроигл показан на рисунке  \ref{ris:Figures/struct1.png}., который включает в себя три части: рабочий электрод (WE), противоэлектрод (CE) и электрод сравнения (RE). Массив микроигл изготовлен из нержавеющей стали SUS316L с использованием процесса штамповки металла. Каждая микроигла имеет длину 1 мм, ширину 0,25 мм и толщину 0,1 мм. Затем противоэлектрод покрывают золотом, рабочий электрод покрывают золотом и ПАНИ, а электрод сравнения покрывают Ag/AgCl. Каждый массив WE имеет площадь 3 мм×3 мм и состоит из 3×4 микроиглы. Рабочая площадь каждого массива микроигл составляет около 1,2 мм.2. Все иглы скошены для предотвращения нестабильности сигнала из-за отслоения глюкозооксидазы при прокалывании под кожей


% \imghh{150mm}{Figures/struct1.png}{Структура датчика матрицы микроигл;}


Структурная схема устройства представлена на рисунке \ref{ris:Figures/struct.png}.
\imghh{150mm}{Figures/struct.png}{Структурная схема устройства}


Принцип работы устройства заключается в следующем: через электроды сигнал передается на усилитель биопотенциалов, который усиливает амплитуду сигнала. Эти данные поступают в микроконтроллер, где проходят первичную обработку, и с помощью алгоритма на языке Си анализируются. Помимо анализа, данные передаются по модулю Bluetooth, интегрированному в микроконтроллер, и записываются во внешнюю Flash-память.



\end{sloppypar}
% это раздел структурной схемы
\begin{sloppypar} % помогает в кириллическом документе выровнять текст по краям
\newpage % Так добавляется  новая страница
\section{РАЗРАБОТКА ПРИНЦИПИАЛЬНОЙ СХЕМЫ УСТРОЙСТВА} %Объявили начало раздела
Электрическая принципиальная схема представлена в приложении.

\subsection{Выбор акселерометра}
Стоит выяснить, как работают и устроены акселерометры. \cite{MEMS} Это датчики движения, входным сигналом которых является скорость и ускорение объекта. Отличительной особенностью данных устройств является их компактность и стоимость за счет налаженного производства микроэлектромеханических систем (МЭМС).

Основное применение датчики движения нашли в промышленности, а именно в авиации для определения положения летающего аппарата в пространстве и в строительстве. В медицине датчики движения используются редко, однако некоторые методики включают использование акселерометров.

Современные МЭМ акселерометры разделяют по физическому принципу детектирования ускорений, однако широкое распространение получили только 3 вида:

\begin{itemize}

	\item[--]Пьезоэлектрические, основой которых является пъезокристалл. Деформации кристалл приводят к появлению на нем разности потенциалов. Такие акселерометры имеют широкий диапазон частот и выдерживают значительные нагрузки. Однако пьезоэффект возникает только в момент деформации, что не позволяет измерять статические ускорения наподобие гравитационного. Также пьезоэлектрические акселерометры из-за значительного сопротивления пьезокристалла и малой разности потенциалов при деформации требуют высокоомного соединения со схемой. 

	\item[--]Пьезорезистивные своими характеристиками не сильно отличаются от ПЭА, имея столь же малую термостабильность и стабильность смещения. Однако получение полезного электрического сигнала происходит на мостовой схеме с пьезорезистивными элементами, при этом нет необходимости использования высокоомного подключения. Также присутствует возможность самотестирования акселерометров и измерения статических нагрузок 

	\item[--]Емкостные – самый распространенный вид МЭМ акселерометров. Принцип действия заключается в измерении реакции измерительной ячейки, состоящей из сложного конденсатора с переменной емкостью на зондирующий сигнал. При измерении ускорения инерционная масса двигает нестатичную обкладку конденсатора, вследствие чего меняется емкость. При этом емкостные конденсаторы не имеют проблем, связанных с природой пьезоэффекта, а именно имеют конструкторскую легкость при подключении в цепь и возможность самотестирования. Также основными преимуществами является высокая термостабильность. Недостатком можно назвать сложность конструкции, однако при налаженном производстве это фактор не оказывает значительного влияния.

\end{itemize}

Таким образом, современные малогабаритные измерительные модули целесообразно конструировать с емкостными акселерометрами, за счет их стабильности отсутствия требований в схемах высокоомного подключения.


Согласно техническому заданию нам необходим акселерометр с диапазоном регистрируемых ускорений от 2g до 8g и возможностью выдачи показаний с частотой 400 Гц. Данным требованиям соответствует 3-осевой акселерометр ADXL345 \cite {ADXL}, его основные характеристики представлены ниже.


\begin{onehalfspace}
\begin{itemize}
	\item[--] Тип датчика: цифровой, емкостной;
	\item[--]Диапазон регистрируемых ускорений ±2g, ±4g, ±8g, ±16g;
	\item[--]Частота обновления показаний: задается пользователем в диапазоне 0.1-3200 Гц;
	\item[--]Сверхнизкое потребление:  23 мкA в режиме преобразования и 0.1 мкA в режиме ожидания;
	\item[--] напряжение питания: 2-3.6В;
	\item[--] Интерфейс цифрового вывода: $I^2$C, SPI; 
	\item[--] Разрядность: настраиваемая пользователем -- 10 бит в диапазоне ±2g, 13 бит в остальных диапазонах.	
\end{itemize}
\end{onehalfspace}




Структурная схема акселерометра из даташита ADXL345 приведена на рисунке \ref{ris:Figures/accel.png}.

\imghh{160mm}{Figures/accel.png}{Структурная схема акселерометра}

Видно, что устройство состоит из 3-осевого ''сенсора'',  представляющего собой несколько конденсаторов с нестатичными обкладками, ''чувствительной электроники'', аналого-цифрового преобразователя, цифрового фильтра, буфера FIFO для временного хранения результатов преобразования, контроллера питания и логического устройства, контролирующего работу акселерометра и логику прерываний. Устройство содержит выводы данных, соответствующие интерфейсам $I^2$C и SPI. Для связи с акселерометром мы будем использовать $I^2$C. Схема подключения представлена на рисунке \ref{ris:Figures/i2c.png}.


\imghh{100mm}{Figures/i2c.png}{Cхема подключения акселерометра к микроконтроллеру по $I^2$C}


Как видно из рисунка \ref{ris:Figures/i2c.png}, для активации интерфейса $I^2$C необходимо подтянуть вывод $\overline{CS}$ к питанию.

Так же, в даташите приведена рекомендованная для минимизации шумов схема включения акселерометра(рисунок \ref{ris:Figures/accel_connect.png}).

\imghh{100mm}{Figures/accel_connect.png}{Типовая схема включения акселерометра}
Нумерация и назначение выводов ADXL345 приведено ниже (рисунки \ref{ris:Figures/accel_io.png}, \ref{ris:Figures/accel_io2.png}).
\imghh{90mm}{Figures/accel_io.png}{Нумерация выводов}
\imghh{160mm}{Figures/accel_io2.png}{Назначение выводов}




\subsection{Выбор микроконтроллера}
С учетом технического задания микроконтроллер должен обладать следующими свойствами:

\begin{onehalfspace}
	\begin{itemize}
		\item[--]Интерфейс для работы с микросхемой акселерометра: SPI или $I^2$C;
		\item[--]Для передачи данных по Bluetooth: встроенный стек протокола Bluetooth;
		\item[--]Малое энергопотребление;
		\item[--]Свободные выводы для подключения индикатора и выводов прерываний от акселерометра;
	\end{itemize}
\end{onehalfspace}

Для решения задачи был выбран микроконтроллер STM32WB35CCU6A фирмы ST Microelectronics \cite {STM}.STM32WB35 содержит два производительных ядра ARM-Cortex:
\begin{onehalfspace}
	\begin{itemize}
		\item[--] ядро ARM® -Cortex® M4 (прикладное), работающее на частотах до 64 МГц, для пользовательских задач имеется модуль управления памятью, модуль плавающей точки, инструкции ЦОС (цифровой обработки сигналов), графический ускоритель (ART accelerator);
		\item[--] ядро ARM®-Cortex® M0+ (радиоконтроллер) с тактовой частотой 32 МГц, управляющее радиотрактом и реализующее низкоуровневые функции сетевых протоколов;
	\end{itemize}
\end{onehalfspace}
Данный микроконтроллер включает в себя все необходимые периферийные устройства, такие как интерфейсы передачи данных $I^2$C,необходимый для подключения к акселерометру, и радиомодуль с поддержкой Bluetooth, диапазон питающего напряжения от 2 до 3,6 В. 
Основные характеристики:
\begin{onehalfspace}
	\begin{itemize}
		\item[--] типовое энергопотребление 50 мкА/МГц (при напряжении питания 3 В);
		\item[--] потребление в режиме останова 1,8 мкА (радиочасть в режиме ожидания (standby));
		\item[--] потребление в выключенном состоянии (Shutdown) менее 50 нА;
		\item[--] диапазон допустимых напряжений питания 1,7…3,6 В (встроенный DC-DC–преобразователь и LDO-стабилизатор);
		\item[--] рабочий температурный диапазон -40…105°С.
	\end{itemize}
\end{onehalfspace}
Структурная схема микроконтроллера приведена на рисунке \ref{ris:Figures/stm32.png}, а назначение выводов портов корпуса на рисунке \ref{ris:Figures/stm32io.png}

\imghh{160mm}{Figures/stm32.png}{Структурная схема}
\imghh{100mm}{Figures/stm32io.png}{Назначение выводов}

Подключение будет осуществляться согласно типовой схеме из Application note\cite {STM_an}(рисунок \ref{ris:Figures/stm32an.png})
\imghh{160mm}{Figures/stm32an.png}{Типовая схема подключения STM32WB35}



\subsection{Блок питания}

Блок питания будет состоять из аккумулятора LP-130-232635 \cite {li-pol} и DC-DC преобразователя LM3671 \cite {dc-dc}.
Аккумулятор литий-полимерный LP-130-232635 имеет номинальную емкость 130 мАч, номинальное напряжение 3,7 В, вес 3г. Длина: 35±1 мм. Ширина: 26±1 мм. Толщина: 2,3±1 мм. 



DC-DC преобразователь LM3671MF с фиксированным выходным напряжением 3,3 В. Типичный ток покоя 16 мкA, типичный ток в выключенном состоянии - 0.01 мкA, максимальная нагрузка по току 600 мА.

Подключение DC-DC преобразователя будет будет осуществляться согласно типовой схеме из Data Sheet \cite {dc-dc} (рисунок \ref{ris:Figures/dc-dc.png})
\imghh{160mm}{Figures/dc-dc.png}{Типовая схема включения DC-DC–преобразователя}











\end{sloppypar}
% это раздел принципиальной схемы
\begin{sloppypar} % помогает в кириллическом документе выровнять текст по краям
\newpage % Так добавляется  новая страница
\section{РАЗРАБОТКА ПРОГРАММЫ} %Объявили начало раздела
\subsection{Разработка алгоритма}
Сначала нужно понять, что необходимо для выполнения задач, поставленных в задании. Нужно проиницилизировать необходимую переферию и настроить их в нужный режим работы, для чего нужно тщательно изучить даташиты на все компоненты.

Требуется обновлять данные с частотой 400Гц --  для этого нужен таймер. Нужно каким-то образом обрабатывать полученные данные и отсылать результат по Bluetooth. 

Для работы программы необходимо для начала разработать алгоритм. Алгоритм нашего устройства представлен на рисунке \ref{ris:prog/algo.png}.
\imghh{150mm}{prog/algo.png}{Алгоритм работы устройства}


Главное тело программы работает так -  инициализирует всю необходимую переферию микроконтроллера, после чего проверяет launch\_flg - флаг, который поднимается в прерывании от таймера каждые 2.5мс, что соответствует частоте обновления данных в 400Гц. 


Так же есть три прерывания -- по переполнению счетчика таймера, и два внешних - по изменению уровня на выводах акселерометра INT0 INT1, подключенных к выводам микроконтроллера PA4 и PA5.


Если флаг запуска  launch\_flg==true, то ждем, когда установится флаг data\_flg(который устанавливается по готовности данных в акселерометре). После этого проверяем флаг свободного падения free\_fall\_flg==true. Если он активен - то помимо данных об ускорении пишем еще и сообщение о том, что произошло падение. Если free\_fall\_flg==false, то просто передаем данные по ускорению. Таким образом, для определения характера движения была использована гибкая система прерываний ADXL345. Подробнее о ней в следующем разделе.

\subsection{Разработка кода}

\subsubsection{Выбор программного обеспечения}
Для разработки ПО под STM32 можно использовать различные IDE. Самые популярные — IAR, Keil, Coocox (Eclipse). Мы же пойдем по пути, который с недавних пор абсолютно бесплатно и в полном объеме предоставляет сама ST.


STM32CubeIDE – многофункциональное средство разработки, являющееся частью экосистемы STM32Cube от компании STMicroelectronics.
STM32CubeIDE – платформа разработки C/C++ с IP-конфигурацией, генерацией и компиляцией кода и способностью прошивки микроконтроллеров STM32.
Программное обеспечение построено на платформе ECLIPSE™/CDT и пакетов программ GCC для разработки, а также отладчика GDB для прошивки микроконтроллера.


Какие плюсы у данного ПО: абсолютно бесплатно, нет ограничения по размеру кода, есть неплохой отладчик, простая установка и настройка. Так же, стоит отметить, что данная платформа кроссплатформенная - есть версии для Windows, Linux и даже MacOS. Ознакомиться с STM32CubeIDE можно в \cite{STM32CubeIDE}

\subsubsection{Инициализация переферии}
В STM32CubeIDE встроен STM32CubeMx -- программный продукт, позволяющий легко и непринужденно при помощи достаточно понятного графического интерфейса произвести настройку любой имеющейся на борту микроконтроллера периферии. 

Предыстория создания CubeMx такова - ST имеют очень разнообразную линейку микроконтроллеров, тут и Cortex-M0, и Cortex-M0+, и Cortex-M3, и Cortex-M4. Соответственно, встает вопрос о каком-то едином наборе библиотек и едином инструменте для инициализации и конфигурирования всего этого многообразия. Вот для решения этих целей и был выпущен STM32CubeMx.

Суть концепции такова - создаем проект, выбираем микроконтроллер, и нам сразу же предлагается схема со всеми выводами выбранного контроллера. Нажимая на выводы и заходя в разнообразные меню, мы легко настраиваем как периферию, так и режимы работы каждого конкретного вывода. Сразу же очевидные плюсы - можно наглядно увидеть, какие выводы уже заняты, а какие еще свободны (в крупных проектах - более чем полезная функция).
Подробнее об этом можно прочитать в \cite{cube}

В нашем случае нужно создать проект, выбрать микроконтроллер и подключить переферию - таймер, $I^2$C, тактирование. Все это настраивается в графическом интерфейсе.

 Сначала в настройках Reset and Clock Controller(RCC) подключаем кварцевые резонаторы, как показано на рисунке \ref{ris:cub/rcc.png}.
\imghh{150mm}{cub/rcc.png}{Настройки RCC}

Затем подключим интерфейс $I^2$C, как показано на рисунке \ref{ris:cub/i2c.png}.
\imghh{150mm}{cub/i2c.png}{Настройки $I^2$C}

Затем подключим порты ввода-вывода и настроим их как внешний источник прерываний, как показано на рисунке \ref{ris:cub/gpio.png}.
\imghh{150mm}{cub/gpio.png}{Настройки портов ввода-вывода}

После этого можно настроить тактирование на вкладке Clock Configuration, как показано на рисунке \ref{ris:cub/clock.png}.
\imghh{150mm}{cub/clock.png}{Настройки тактирования}

Для включения стека Bluetooth необходимо активировать Inter-Process Communication Controller(IPCCC), Hardware Semaphore (HSEM) (необходим для синхронизации процессов, запущеных на разных ядрах), как показано на рисунках  \ref{ris:cub/IPCC.png}, \ref{ris:cub/HSEM.png}, включить Radio System(RF),как показано на рисунке \ref{ris:cub/RF.png}.

\imghh{150mm}{cub/IPCC.png}{Активируем IPCC}
\imghh{150mm}{cub/HSEM.png}{Активируем HSEM}
\imghh{150mm}{cub/RF.png}{Активируем RF}
Теперь разблокирована вкладка STM32WPAN, где нужно включить стек Bluetooth. В настройках указываем, что конечное устройство будет являться сервером(то есть транслировать данные другим устройствам). Настройки показаны на рисунке \ref{ris:cub/BLE.png}.
\imghh{150mm}{cub/BLE.png}{Активируем Bluetooth}


























\subsubsection{Инициализация ADXL345}
 Карта регистров представлена на рисунке \ref{ris:prog/adxl_regmap.png}.
\imghh{150mm}{prog/adxl_regmap.png}{Описания регистров и их адреса}

Для удобства перенесем эту карту в код: 
\begin{minted} [
% frame=lines,%линия сверху и снизу блока кода
% framesep=15mm, % отступ между линией и кодом
baselinestretch=1, %интервал междустрочный
 % bgcolor=LightGray, %цвет фона
fontsize=\footnotesize, %размер шрифта
% linenos%нумерация строк
]{C}
#define DEVID_ID			0x00
#define THRESH_TAP			0x1D
#define OFSX 			    0x1E
#define OFSY 				0x0F
#define OFSZ 				0x20
#define DUR 				0x21
#define Latent 				0x22
#define Window 				0x23
#define THRESH_ACT			0x24
#define THRESH_INACT		0x25
#define TIME_INACT			0x26
#define ACT_INACT_CTL		0x27
#define THRESH_FF			0x28
#define TIME_FF 			0x29
#define TAP_AXES			0x2A
#define ACT_TAP_STATUS		0x2B
#define BW_RATE 			0x2C
#define POWER_CTL			0x2D
#define INT_ENABLE			0x2E
#define INT_MAP 			0x2F
#define INT_SOURCE			0x30
#define DATA_FORMAT			0x31
#define DATAX0   			0x32
#define DATAX1   			0x33
#define DATAY0   			0x34
#define DATAY1   			0x35
#define DATAZ0   			0x36
#define DATAZ1   			0x37
#define FIFO_CTL  			0x38
#define FIFO_STATUS  		0x39
\end{minted}

Далее необходимо эти регистры проинициализировать.
Связь с акселерометром происходит по $I^2$C. Значит, перед тем как передать настройки регистров, нужно сформировать необходимый пакет и отправить его на шину $I^2$C.  Для этого объявлена функция adxl\_write.


\begin{minted} [
% frame=lines,%линия сверху и снизу блока кода
% framesep=15mm, % отступ между линией и кодом
baselinestretch=1, %интервал междустрочный
 % bgcolor=LightGray, %цвет фона
fontsize=\footnotesize, %размер шрифта
% linenos%нумерация строк
]{C}
void adxl_write(uint8_t address_reg, uint8_t value) {//запись в регистр
	uint8_t data[2];//массив для хранения посылки
	data[0] = address_reg;//cначала передаем адрес регистра в который будем читать
	data[1] = value; //Затем значение которое нужно записываем
	HAL_I2C_Master_Transmit(&hi2c1, adxl_addr, data, 2, timeout)  ;//отправляем массив с адресом и значением
}

\end{minted}


Так же для инициализации объявлена функция adxl\_init.

\begin{minted} [
% frame=lines,%линия сверху и снизу блока кода
% framesep=15mm, % отступ между линией и кодом
baselinestretch=1, %интервал междустрочный
 % bgcolor=LightGray, %цвет фона
fontsize=\footnotesize, %размер шрифта
 % linenos%нумерация строк
]{C}
void adxl_init(void) {//инициализация акселерометра и настройка

	adxl_write(DATA_FORMAT, RANGE_8G);  //настраиваем диапазон +- 8g

	adxl_write(POWER_CTL, 0x00);  // выход из режима сна
	adxl_write(POWER_CTL, 0x08);  //включаем преобразование
	
	adxl_write(THRESH_FF, 0x06);  //настраиваем значение free fall treshold = 62.5mg*8=0.5g
	adxl_write(TIME_FF, 0x02);     //настраиваем время free fall
	//по двум параметрам выше срабатывает прерывание

	adxl_write(INT_ENABLE, FREE_FALL);     //включаем прерывание от free fall
	adxl_write(INT_MAP, FREE_FALL);  //назначаем его на вывод IN1

	adxl_write(INT_ENABLE, DATA_READY); //включаем прерывание по готовности данных
	adxl_write(INT_MAP, DATA_READY);  //назначаем его на вывод IN0

}

\end{minted}

Разберем строки подробнее.

\begin{minted} [
% frame=lines,%линия сверху и снизу блока кода
% framesep=15mm, % отступ между линией и кодом
baselinestretch=1, %интервал междустрочный
 % bgcolor=LightGray, %цвет фона
fontsize=\footnotesize, %размер шрифта
 % linenos%нумерация строк
]{C}
void adxl_init(void) {//инициализация акселерометра и настройка

	adxl_write(DATA_FORMAT, RANGE_8G);  //настраиваем диапазон +- 8g
}
\end{minted}


Эта строка записывает в регистр DATA\_FORMAT значение, соотвествующее диапазону +-8g. Возможные диапазоны были так же предварительно объявлены. 


\begin{minted} [
% frame=lines,%линия сверху и снизу блока кода
% framesep=15mm, % отступ между линией и кодом
baselinestretch=1, %интервал междустрочный
 % bgcolor=LightGray, %цвет фона
fontsize=\footnotesize, %размер шрифта
 linenos%нумерация строк
]{C}
void adxl_init(void) {
//инициализация акселерометра и настройка
#define RANGE_2G		0x00
#define RANGE_4G		0x01
#define RANGE_8G		0x02
#define RANGE_16G		0x03
}
\end{minted}
 
Здесь и далее вся информация из \cite{ADXL}, на рисунках \ref{ris:prog/range.png} и \ref{ris:prog/range1.png} представлены необходимые фрагменты.
\imghh{70mm}{prog/range.png}{Описаниe регистра }
\imghh{70mm}{prog/range1.png}{Возможные значения для настройки}


\begin{minted} [
% frame=lines,%линия сверху и снизу блока кода
% framesep=15mm, % отступ между линией и кодом
baselinestretch=1, %интервал междустрочный
 % bgcolor=LightGray, %цвет фона
fontsize=\footnotesize, %размер шрифта
 % linenos%нумерация строк
]{C}

	adxl_write(POWER_CTL, 0x00);  // выход из режима сна
	adxl_write(POWER_CTL, 0x08);  //включаем преобразование

\end{minted}

Данные две строчки сначала очищают регистр POWER\_CTL, затем инициализируют включение режима преобразования. Регистр показан на рисунке \ref{ris:prog/pwrctl.png} 
\imghh{70mm}{prog/pwrctl.png}{регистр POWER\_CTL}


\begin{minted} [
% frame=lines,%линия сверху и снизу блока кода
% framesep=15mm, % отступ между линией и кодом
baselinestretch=1, %интервал междустрочный
 % bgcolor=LightGray, %цвет фона
fontsize=\footnotesize, %размер шрифта
 % linenos%нумерация строк
]{C}

	adxl_write(THRESH_FF, 0x06);  //настраиваем значение free fall treshold = 62.5mg*8=0.5g
	adxl_write(TIME_FF, 0x14);     //настраиваем время free fall =5ms*30=150ms
	//по двум параметрам выше срабатывает прерывание
\end{minted}

Данные две строчки записывают значения в регистры THRESH\_FF  и TIME\_FF  значения ускорения и времени, необходимых для того чтобы сработало прерывание по свободному падению. На рисунке \ref{ris:prog/tresh.png} показано описание регистров.
\imghh{70mm}{prog/tresh.png}{Описаниe регистров THRESH\_FF  и TIME\_FF }



\begin{minted} [
% frame=lines,%линия сверху и снизу блока кода
% framesep=15mm, % отступ между линией и кодом
baselinestretch=1, %интервал междустрочный
 % bgcolor=LightGray, %цвет фона
fontsize=\footnotesize, %размер шрифта
 % linenos%нумерация строк
]{C}


	adxl_write(INT_ENABLE, FREE_FALL);     //включаем прерывание от free fall
	adxl_write(INT_MAP, FREE_FALL);  //назначаем его на вывод IN1

	adxl_write(INT_ENABLE, DATA_READY); //включаем прерывание по готовности данных
	adxl_write(INT_MAP, DATA_READY);  //назначаем его на вывод IN0

\end{minted}

Эти строки включают прерывания  по свободному падению и готовности данных и назначают их на соотвествующие выводы. Возможные для настройки прерывания показан на рисунке \ref{ris:prog/interrupt.png}, про активацию необходимых прерываний и настройку выводов показано на рисунке \ref{ris:prog/interrupt_map.png}
 \imghh{150mm}{prog/interrupt.png}{Возможные прерывания}
 \imghh{70mm}{prog/interrupt_map.png}{Настройка прерываний}
 
 
 На этом инициализацию устройства можно считать законченной.
 
 Так же объявляется функция adxl\_read для чтения данных с регистров акселерометра.
 
 \begin{minted} [
% frame=lines,%линия сверху и снизу блока кода
% framesep=15mm, % отступ между линией и кодом
baselinestretch=1, %интервал междустрочный
 % bgcolor=LightGray, %цвет фона
fontsize=\footnotesize, %размер шрифта
 % linenos%нумерация строк
]{C++}
uint8_t adxl_read(uint8_t address_reg) {//чтение с регистра одного байта
	address_reg |= 0x80;  // маска для задания бита чтения
	uint8_t data[1]={0}; //переменная для прочитанных данных

	HAL_I2C_Master_Transmit(&hi2c1, adxl_addr, address_reg, 1, timeout);
	//посылаем адрес регистра, с которого хотим читать

	HAL_I2C_Master_Receive(&hi2c1, adxl_addr, data, 1, timeout);
	//читаем байт в переменную data
	return data;
}

\end{minted}
 
 
 
Данный фрагмент сначала формирует адрес и по маске устанавливает бит, отвечающий за чтение, согласно интерфейсу $I^2$C. Затем объявляется переменная для хранения данных. 

 \begin{minted} [
% frame=lines,%линия сверху и снизу блока кода
% framesep=15mm, % отступ между линией и кодом
baselinestretch=1, %интервал междустрочный
 % bgcolor=LightGray, %цвет фона
fontsize=\footnotesize, %размер шрифта
 % linenos%нумерация строк
]{C++}
HAL_I2C_Master_Transmit(&hi2c1, adxl_addr, address_reg, 1, timeout);
\end{minted}
 передает на шину $I^2$C адрес устройства с которым хочет связаться в формате чтения, и передает один байт, в котором содержится адрес регистра.
 
 Далее функция
    \begin{minted} [
% frame=lines,%линия сверху и снизу блока кода
% framesep=15mm, % отступ между линией и кодом
baselinestretch=1, %интервал междустрочный
 % bgcolor=LightGray, %цвет фона
fontsize=\footnotesize, %размер шрифта
 % linenos%нумерация строк
]{C++}
   HAL_I2C_Master_Receive(&hi2c1, adxl_addr, data, 1, timeout);
\end{minted}
запрашивает у устройства один байт данных, и записывает его в переменную, которую функция потом вернет.




\end{sloppypar}
% это раздел принципиальной схемы
\begin{sloppypar} % помогает в кириллическом документе выровнять текст по краям
\newpage % Так добавляется  новая страница

\section*{ЗАКЛЮЧЕНИЕ} %Объявили начало раздела
\addcontentsline{toc}{section}{ЗАКЛЮЧЕНИЕ}

\end{sloppypar}
% это заключение
\newpage
\renewcommand\refname{\centering СПИСОК ИСПОЛЬЗОВАННЫХ ИСТОЧНИКОВ}
\begin {thebibliography} {99}
\addcontentsline{toc}{section}{СПИСОК ИСПОЛЬЗОВАННЫХ ИСТОЧНИКОВ}

\bibitem {ADS1293}Беляев, А. О. Анализ аналоговых характеристик микросхемы ADS1293 для применения в медицинской технике / А. О. Беляев, В. В. Кириенко // Инженерный вестник Дона. – 2014. – № 3(30). – С. 67. – EDN TFXFTD

\bibitem {DS1293}
Data Sheet на AFE микросхему ADS1293  [Электронный ресурс]. URL:\href{https://radioaktiv.ru/ds/ti/snas602b.pdf}{https://radioaktiv.ru/ds/ti/snas602b.pdf} (Дата обращения: 6.05.2023)



\bibitem {STM}
Data Sheet на микроконтроллер STM32WB55CCU6  [Электронный ресурс]. URL:\href{https://www.st.com/resource/en/datasheet/stm32wb55cc.pdf}{https://www.st.com/resource/en/datasheet/stm32wb55cc.pdf} (Дата обращения: 10.05.2023)


\bibitem {STM_an} 
Application note на микроконтроллеры серии STM32WB  [Электронный ресурс]. URL:\href{https://www.st.com/resource/en/application_note/an5165-development-of-rf-hardware-using-stm32wb-microcontrollers-stmicroelectronics.pdf}{https://www.st.com/resource/en/application\_note/an5165-development-of-rf-hardware-using-stm32wb-microcontrollers-stmicroelectronics.pdf} (Дата обращения: 11.05.2023)


\bibitem {li-pol}
Спецификация на Li-pol аккумулятор LP-310-233350  [Электронный ресурс]. URL:\href{https://static.chipdip.ru/lib/412/DOC005412828.pdf}{https://static.chipdip.ru/lib/412/DOC005412828.pdf} (Дата обращения: 11.05.2023)



\end {thebibliography}




% это список использованных источников
 
% \input{append}% это приложение

% \input{RefProject-base}% осн часть


%\begin{landscape}
% текст в альбомной ориентации
% (таблица, рисунок, схема и т. п.)
%\end{landscape}

% \likechapter{Вступление}
% ебал все в рот



%\input{RefProject-App}% приложение








\end{document}