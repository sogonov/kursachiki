\documentclass[a4paper]{article} %размер бумаги устанавливаем А4
% \usepackage[T2A]{fontenc}
\usepackage{polyglossia}
\setmainlanguage[babelshorthands=true]{russian} % Язык по-умолчанию русский с поддержкой приятных команд пакета babel
\setotherlanguage{english} % Дополнительный язык = английский (в американской вариации по-умолчанию)
\newfontfamily{\cyrillicfonttt}{Times New Roman}
% \newfontfamily\cyrillicfont{Times New Roman}
\usepackage{minted} %пакет для подсветки кода
\usepackage{color}
\usepackage{xcolor} % to access the named colour LightGray
\definecolor{LightGray}{gray}{0.9}

% для отступа в первом абзаце
\usepackage{indentfirst} 
\parindent=1.25cm % длина отступа в абзацах
% для продвинутых списков
\usepackage{enumitem} 
\usepackage{amssymb,amsfonts,amsmath,cite,enumerate,float,indentfirst} %пакеты расширений
\usepackage{graphicx}% для вставки картинок
\usepackage{url} % добавляем поддержку url-ссылок
\usepackage{hyperref} % пакет для интеграции гиперссылок
\usepackage{amsmath} % добавляем поддержку математических символов
\usepackage{multirow} % понадобится для создания таблицы с объединенными строками
\usepackage{pdfpages}% Добавление внешних pdf файлов
\usepackage{tocloft}
\renewcommand{\cftsecfont}{\mdseries}
\renewcommand{\cftsecpagefont}{\mdseries}
\cftsetindents{section}{0em}{2em}
\cftsetindents{subsection}{0em}{3em}
\cftsetindents{subsubsection}{0em}{4em}
\renewcommand\cfttoctitlefont{\hfill\normalsize\mdseries}
\renewcommand\cftaftertoctitle{\hfill\mbox{}}
\renewcommand{\cftsecleader}{\cftdotfill{\cftdotsep}}

\usepackage{fontspec}
\setmainfont{Times New Roman} %шрифт 
\graphicspath{{images/}}%путь к рисункам
\usepackage[14pt]{extsizes} % для того чтобы задать нестандартный 14-ый размер шрифта

%\makeatletter
%\renewcommand{\@biblabel}[1]{#1.} % Заменяем библиографию с квадратных скобок на точку:
%\makeatother


\usepackage[tableposition=top]{caption}
\usepackage{subcaption}
\DeclareCaptionLabelFormat{gostfigure}{Рисунок #2}
\DeclareCaptionLabelFormat{gosttable}{Таблица #2}
\DeclareCaptionLabelSeparator{gost}{~–~}
\captionsetup{labelsep=gost}
\captionsetup[figure]{labelformat=gostfigure}
\captionsetup[table]{labelformat=gosttable}
\renewcommand{\thesubfigure}{\asbuk{subfigure}}


\makeatletter

\renewcommand{\section}{\@startsection{section}{1}{0pt}%
                                {-3.5ex plus -1ex minus -.2ex}%
                                {2.3ex plus .2ex}%
{\centering\hyphenpenalty=10000\normalfont\normalsize\mdseries}}

\renewcommand{\subsection}{\@startsection{subsection}{1}{0pt}%
                                {-3.5ex plus -1ex minus -.2ex}%
                                {2.3ex plus .2ex}%
{\centering\hyphenpenalty=10000\normalfont\normalsize\mdseries}}
\renewcommand{\subsubsection}{\@startsection{subsubsection}{1}{0pt}%
                                {-3.5ex plus -1ex minus -.2ex}%
                                {2.3ex plus .2ex}%
{\centering\hyphenpenalty=10000\normalfont\normalsize\mdseries}}
\makeatother


%\renewcommand{\bibname}{Список использованных источников}
%\addcontentsline{toc}{chapter}{Список использованных источников}
% \usepackage{pdflscape}

\usepackage{setspace}
% выравнивание по ширине
\sloppy



%%% Поля и разметка страницы %%%
\usepackage{pdflscape}  % Для включения альбомных страниц
\usepackage{geometry}   % Для последующего задания полей


\geometry{left=3cm}% левое поле
\geometry{right=1.5cm}% правое поле
\geometry{top=2cm}% верхнее поле
\geometry{bottom=2cm}% нижнее поле

\renewcommand{\theenumi}{\arabic{enumi}}% Меняем везде перечисления на цифра.цифра
\renewcommand{\labelenumi}{\arabic{enumi}}% Меняем везде перечисления на цифра.цифра
\renewcommand{\theenumii}{.\arabic{enumii}}% Меняем везде перечисления на цифра.цифра
\renewcommand{\labelenumii}{\arabic{enumi}.\arabic{enumii}.}% Меняем везде перечисления на цифра.цифра
\renewcommand{\theenumiii}{.\arabic{enumiii}}% Меняем везде перечисления на цифра.цифра
\renewcommand{\labelenumiii}{\arabic{enumi}.\arabic{enumii}.\arabic{enumiii}.}% Меняем везде перечисления на цифра.цифра
\usepackage{lastpage}



\newcommand{\imgh}[3]{\begin{figure}[h]\center{\includegraphics[width=#1]{#2}}\caption{#3}\label{ris:#2}\end{figure}}
\newcommand{\imghh}[3]{\begin{figure}[H]\center{\includegraphics[width=#1]{#2}}\caption{#3}\label{ris:#2}\end{figure}}
\usepackage{totcount}
% \newtotcounter{foofigure}

% \makeatletter
% \renewenvironment{figure}[1][\fps@figure]{
  % \edef\@tempa{\noexpand\@float{figure}[#1]}
  % \@tempa
  % \addtocounter{foofigure}{1}
% }{
  % \end@float
% }
% \makeatother



\newcounter{totfigures}
\newcounter{tottables}
\newcounter{totreferences}

\makeatletter
    \AtEndDocument{%
      \addtocounter{totfigures}{\value{figure}}%
      \addtocounter{tottables}{\value{table}}%
	  
      \immediate\write\@mainaux{%
        \string\gdef\string\totfig{\number\value{totfigures}}%
        \string\gdef\string\tottab{\number\value{tottables}}%   

      }%
    }
\makeatother

	

\newcommand{\empline}{\mbox{}\newline}
\newcommand{\likechapterheading}[1]{ 
    \begin{center}
   \MakeUppercase{#1}
    \end{center}
    \empline}



\makeatletter
    \renewcommand{\@dotsep}{2}
    \newcommand{\l@likechapter}[2]{{\@dottedtocline{0}{0pt}{0pt}{#1}{#2}}}
\makeatother
\newcommand{\likechapter}[1]{    
    \likechapterheading{#1}    
    \addcontentsline{toc}{likechapter}{\MakeUppercase{#1}}}




\usepackage{cite} % Красивые ссылки на литературу


%% Список литературы с красной строки (без висячего отступа) %%%
\patchcmd{\thebibliography} %может потребовать включения пакета etoolbox
 {\advance\leftmargin\labelsep}
 {\leftmargin=0pt%
  \setlength{\labelsep}{\widthof{\ }}% Управляет длиной отступа после точки
  \itemindent=\parindent%
  \addtolength{\itemindent}{\labelwidth}% Сдвигаем правее на величину номера с точкой
  \advance\itemindent\labelsep%
 }
 {}{}


\makeatletter
\def\@biblabel#1{#1 }
\makeatother


\begin{document}

\begin{titlepage}
\newpage
\doublespacing
%\linespread{1.3} % полуторный интервал
%\setlength\parindent{1.25cm}
\begin{center}
ФЕДЕРАЛЬНОЕ ГОСУДАРСТВЕННОЕ АВТОНОМНОЕ\\
ОБРАЗОВАТЕЛЬНОЕ УЧРЕЖДЕНИЕ ВЫСШЕГО ОБРАЗОВАНИЯ\\
«САМАРСКИЙ НАЦИОНАЛЬНЫЙ ИССЛЕДОВАТЕЛЬСКИЙ\\
УНИВЕРСИТЕТ ИМЕНИ АКАДЕМИКА С.П. КОРОЛЕВА»	
 \\
\end{center}

\vspace{5em}

\begin{center}
 Институт информатики и кибернетики \\ 
\end{center}

\begin{center}
Кафедра лазерных и биотехнических систем \\ 
\end{center}


\vspace{3em}

\begin{center}
{Пояснительная записка к курсовому проекту\\''МОНИТОР АКТИВНОСТИ И ОТСЛЕЖИВАНИЯ ПАДЕНИЯ''}
\end{center}

\vspace{11em}




\newbox{\lbox}
\savebox{\lbox}{\hbox{Корнилин Д.В.}}
\newlength{\maxl}
\setlength{\maxl}{\wd\lbox}
\hfill\parbox{15cm}{
\hspace*{5cm}\hspace*{-5cm}Выполнил студент группы 6364-120304D:\hfill\underline{\hspace{4em}}  \hbox to\maxl{Краснов Д.Г.\hfill}\\
\hspace*{5cm}\hspace*{-5cm}Руководитель проекта:\hfill\underline{\hspace{4em}}  \hbox to\maxl{Корнилин Д.В.\hfill }\\
\hspace*{5cm}\hspace*{-5cm}Работа защищена с оценкой:\hfill\underline{\hspace{4em}}  \hbox to\maxl{ \hfill }\\
}


\vspace{\fill}

\begin{center}
Cамара 2023
\end{center}

\end{titlepage}% это титульный лист
\begin{sloppypar} % помогает в кириллическом документе выровнять текст по краям
\newpage % Так добавляется  новая страница
\section*{ЗАДАНИЕ} %Объявили начало раздела

Разработать монитор активности и отслеживания падений  со следующими параметрами: 
\begin{itemize}
    \item[--]	Датчик падений/движения/активности
    \item[--]	Диапазон регистрируемых ускорений от 2g до 8g;
   \item[--]	Частота обновления показаний 400 Гц;
   \item[--]Передача данных по интерфейсу Bluetooth;
   \item[--]	Питание батарейное.
\end{itemize}

\end{sloppypar}
% это задание
\begin{sloppypar} % помогает в кириллическом документе выровнять текст по краям
\newpage % Так добавляется  новая страница
\section*{РЕФЕРАТ} %Объявили начало раздела


% Дипломная работа: \pageref*{LastPage}~страниц, \total{foofigure}~рисунков, источников, 1 приложение
% \tottables~табл...


Дипломная работа: \pageref*{LastPage}~с., \totfig~рис., \tottab~табл...



По теореме Вейерштрасса~\cite {F},
непрерывная на отрезке функция % ...


\end{sloppypar}
% это реферат
\input{cont}% это содержание
\input{intro}% это введение


\input{RefProject-base}% осн часть

 \input{structure}% это раздел структурной схемы
 
% \input{append}% это приложение
\input{biblio}% это приложение




%\begin{landscape}
% текст в альбомной ориентации
% (таблица, рисунок, схема и т. п.)
%\end{landscape}

% \likechapter{Вступление}
% ебал все в рот



%\input{RefProject-App}% приложение








\end{document}